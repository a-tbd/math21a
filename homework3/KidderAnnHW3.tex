\documentclass{article}%
\usepackage[T1]{fontenc}%
\usepackage[utf8]{inputenc}%
\usepackage{lmodern}%
\usepackage{textcomp}%
\usepackage{lastpage}%
%
\title{Homework 3}%
\author{Ann Kidder}%
\date{September 22, 2017}%
%
\begin{document}%
\normalsize%
\maketitle%
\section{Chapter 9}%
\subsection{True/False}%
\begin{enumerate}%
\item%
True. $u\cdot v$ results in a scalar so the order of multiplication does not matter.%
\item%
False. $<1,2,3>\times <4,5,6> = <-3,6,-3>$ However, $<4,5,6> \times <1,2,3> = <3,-6,3>$                     Thus $u \times v = -v \times u$%
\item%
True. Though the signs are opposite for \$u \textbackslash{}times v\$ and \$v \textbackslash{}times u\$, once the components are                     squared they will be the same so the magnitudes are equal.%
\item%
True.  $u \cdot v$ will be a scalar so it does not matter what order $k$ and u and v are multipled.%
\item%
True.  It does not matter what order the scalar is multiplied with the vector.%
\item%
True. $|u \times w|$ is the area of a parallelogram with sides u and w.                      If you connect this to the parallelogram with sides v and w by the shared side w, you will get a total area                     of u+v by w, which is the same as $(u+v) \times w$%
\item%
True.  The final result is a scalar, so it does not matter what order the components are multipled by the cross and dot product.%
\item%
False. The cross product of two vectors is another vector so order of operations matters.                      $u \times (v \times w) = v(u \times w) - w(u \times v)$%
\item%
True.  Because this can be rewritten as $v \cdot (u \times u)$ and $u \times u = 0$ this is true.%
\item%
True.  $(u+v)\times v = u \times v + v \times v$ and $v\times v = 0$ so this is true.%
\item%
True. The cross product of two unit vectors will have a magnitude of one, so it will also be a unit vector.%
\item%
False.  It represents a line only if one and only one coefficient is 0.%
\item%
True.  Because $z^2=0$ the set of all points described is a circle in the xy-plane%
\item%
False.  $u \cdot v = u_1 v_1 + u_2 v_2$, which is a scalar, not a vector.%
\item%
False.  $<1,1,0> \cdot <-1,1,0> = 0$%
\item%
False.  If u and v are non-zero, parallel vectors then $u \times v = 0$  Example, $<1,1,0> \times <2,2,0> = <0,0,0>$%
\item%
True.  If $u \cdot v = 0$ that means either u or v is zero.%
\item%
True. $|u \cdot v| = \sqrt{u_1 v_1^2 + u_2 v_2^2 + u_3 v_3^2}$                     and $||u|| ||v|| = \sqrt{u_1^2+u_2^2+u_3^2}\sqrt{v_1^2+v_2^2+v_3^2}$.  If you square                     both sides and multiply out the right, you see that the highest multiple of the left is 2 while the right is 4%
\end{enumerate}

%
\end{document}